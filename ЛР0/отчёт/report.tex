\documentclass{article}
\usepackage{goststyle} % Подключение созданного стиля
\usepackage{guapTitle}

\usepackage{listings}
\usepackage{color}
\usepackage{graphicx}

\definecolor{codegreen}{rgb}{0,0.6,0}
\definecolor{codegray}{rgb}{0.5,0.5,0.5}
\definecolor{codepurple}{rgb}{0.58,0,0.82}
\definecolor{backcolour}{rgb}{0.95,0.95,0.92}

\lstdefinestyle{mystyle}{
    backgroundcolor=\color{backcolour},
    commentstyle=\color{codegreen},
    keywordstyle=\color{magenta},
    numberstyle=\tiny\color{codegray},
    stringstyle=\color{codepurple},
    basicstyle=\ttfamily\footnotesize,
    breakatwhitespace=false,
    breaklines=true,
    captionpos=b,
    keepspaces=true,
    numbers=left,
    numbersep=5pt,
    showspaces=false,
    showstringspaces=false,
    showtabs=false,
    tabsize=2,
    frame=single,
    escapeinside=``
}

\lstset{style=mystyle}


\begin{document}
\guapTitle{43}{старший преподаватель}{М. Д. Поляк}{Знакомство с системой контроля версий git и сервисом GitHub}{Операционные системы}{М211}{}{М. М. Приходько}{25}
\section*{Текст задания}
\begin{itemize}
    \item Создать файл \texttt{goals.md} с заголовками \texttt{Мой опыт} и \texttt{Мои цели}, добавить абзац об опыте и списки целей и задач.
    \item Создать файл \texttt{info.md} с ФИО, названием ОС и текущими датой и временем.
    \item Использовать Git из командной строки: команды \texttt{git clone}, \texttt{add}, \texttt{commit}, \texttt{push}.
    \item Подготовить отчет в \LaTeX{} и сохранить его как \texttt{report.pdf} вместе с исходным \texttt{.tex}-файлом в репозиторий.
\end{itemize}

\section*{Файл goals.md}
\begin{lstlisting}
### Мой опыт                                   
 Разработал консольное приложение на языке c++ для вычисления трансцендентных уравнений численными методами:
1. Метод биссекций.
2. Метод Ньютона.
3. Метод хорд.
Также разработал приложение с графическим интерфейсом на языке c++ с помощью платформы QT для автопроката.

[OS репозиторий](https://github.com/mirik-cyber/OS)

### Мои цели
- Сдать все лабораторные до мая
- Получить по предмету 5 автоматом
- Удалить Linux
### Шаги для достижения
* Делать лабораторные работы
* Уделять время изучению материала
* Проявить терпение...
\end{lstlisting}
\section*{Файл info.md}
\begin{lstlisting}
    Приходько Мирослав Михайлович
    Windows
    21.03.2025 20:02
\end{lstlisting}
\section*{Вывод}
В ходе выполнения вводной лабораторной работы я познакомился с системой контроля версий Git и сервисом GitHub. Были изучены базовые команды командной строки для работы с репозиторием: клонирование, добавление и фиксация изменений, отправка их на удалённый сервер. Также был освоен синтаксис языка разметки Markdown для оформления текстовых файлов.
\end{document}